\newglossaryentry{API REST}{
	name={API REST},
	description={\gls{API} que sigue los principios arquitectónicos \gls{REST}, permitiendo interacciones con los recursos usando métodos estándar HTTP}
}

\newglossaryentry{API}{
	text={API},
	name={API (\textit{Application Programming Interface})},
	description={Mecanismo que permite la interoperabilidad de aplicaciones heterogéneas, permitiendo a un programa acceder a las funciones y datos de otro}
}

\newglossaryentry{REST}{
	text={REST},
	name={REST (\textit{Representational State Transfer})},
	description={Estilo arquitectónico para sistemas hipermedia distribuidos, como la \textit{World Wide Web}. Se basa en la transferencia de representaciones de recursos, que son identificados por \glslink{URI}{URIs}, y la manipulación de estos recursos mediante métodos estándar de \gls{HTTP}. Para una descripción detallada, leer el \href{https://www.ics.uci.edu/~fielding/pubs/dissertation/rest_arch_style.htm}{quinto capítulo de la tesis doctoral de Roy Fielding}}
}

\newglossaryentry{HTTP}{
	text={HTTP},
	name={HTTP (\textit{Hypertext Transfer Protocol})},
	description={Protocolo de comunicación que permite la transferencia de información en la web, caracterizado por operar sin estado y basado en el modelo cliente-servidor}
}

\newglossaryentry{URI}{
	text={URI},
	name={URI (\textit{Uniform Resource Identifier})},
	description={Cadena de caracteres con estructura fija que identifica un recurso en la web de forma única}
}

\newglossaryentry{JSON}{
	text={JSON},
	name={JSON (\textit{JavaScript Object Notation})},
	description={Notación de objetos de JavaScript, un formato ligero de intercambio de datos que es fácil de leer y escribir para humanos y fácil de analizar y generar para máquinas}
}

\newglossaryentry{sobreexposición}{
  name={sobreexposición},
  description={Fenómeno que ocurre cuando una \gls{API} expone más información de la necesaria, lo que puede llevar a problemas de seguridad y a una mayor complejidad en el mantenimiento}
}

\newglossaryentry{middleware}{
	name={middleware},
	description={Software que actúa como intermediario entre dos aplicaciones o sistemas, permitiendo la comunicación entre ellos}
}

\newglossaryentry{CRUD}{
	text={CRUD},
	name={CRUD (\textit{Create, Read, Update, Delete})},
	description={Conjunto de operaciones básicas que se pueden realizar sobre un recurso en un sistema de información: crear, leer, actualizar y eliminar}
}