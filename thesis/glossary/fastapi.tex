\newglossaryentry{FastAPI}{
	name={FastAPI},
	description={Framework moderno, rápido (de alto rendimiento) y fácil de usar para construir \glslink{API}{APIs} con \gls{Python} basado en las anotaciones de tipo estándar de \gls{Python}, de acuerdo con su \href{https://fastapi.tiangolo.com/}{documentación}}
}

\newglossaryentry{Pydantic}{
	name={Pydantic},
	description={Librería de \gls{Python} que facilita la validación y serialización de datos, especialmente útil para trabajar con datos JSON. Más información en su \href{https://docs.pydantic.dev/latest/}{documentación}}
}

\newglossaryentry{Swagger}{
	text={Swagger},
	name={Swagger UI},
	description={Herramienta que permite visualizar y probar \glslink{API}{APIs} de manera interactiva utilizando documentación generada automáticamente en formato \gls{OpenAPI}. Es especialmente útil para explorar y probar endpoints durante el desarrollo. Más información en su \href{https://swagger.io/tools/swagger-ui/}{documentación}}
}

\newglossaryentry{Redoc}{
	name={Redoc},
	description={Herramienta para renderizar documentación de \glslink{API}{APIs} en formato \gls{OpenAPI} con un diseño limpio y personalizable, orientada a la presentación profesional de la documentación de \glslink{API}{APIs}. Más información en su \href{https://redoc.ly/docs/redoc/}{documentación}}
}

\newglossaryentry{SQLAlchemy}{
	name={SQLAlchemy},
	description={Librería de \gls{Python} que facilita la interacción con bases de datos relacionales mediante un \gls{ORM}. Más información en su \href{https://www.sqlalchemy.org/}{página oficial}}
}

\newglossaryentry{ORM}{
	text={ORM},
	name={ORM (\textit{Object-Relational Mapping})},
	description={Patrón de programación que permite representar objetos como tablas en una base de datos relacional y viceversa, trazando una relación entre las disímiles estructuras de la capa de lógica (orientada a objetos) y la capa de datos (relacional), problema que se conoce como \textit{impedancia del mapeo} (del inglés, \textit{object-relational impedance mismatch})}
}

\newglossaryentry{OpenAPI}{
	name={OpenAPI},
	description={Especificación para describir y documentar \glslink{API}{APIs} de manera estandarizada, permitiendo la generación automática de documentación y la validación de \glslink{API}{APIs}. Más información en su \href{https://spec.openapis.org/oas/latest.html}{especificación oficial}}
}