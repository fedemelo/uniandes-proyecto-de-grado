\newglossaryentry{Python}{
	name={Python},
	description={Lenguaje de programación de propósito general y alto nivel, interpretado y multiparadigma, aunque principalmente orientado a objetos}
}

\newglossaryentry{Jupyter}{
	name={Jupyter},
	description={Software de código abierto que permite crear y compartir documentos interactivos que contienen código, visualizaciones y texto explicativo. Más información en su \href{https://jupyter.org/}{página oficial}}
}

\newglossaryentry{docstring}{
	name={docstring},
	description={Cadena de texto especiales que se escriben al inicio de un módulo, clase, método o función en Python, que describe su funcionalidad y cómo usarlo}
}

\newglossaryentry{reST}{
	text={reST},
	name={reST (reStructuredText)},
	description={Sintaxis de marcado ligero utilizada para documentación, como docstrings en Python. Más información en la \href{https://docutils.sourceforge.io/rst.html}{documentación oficial}}
}

\newglossaryentry{NumPy}{
	name={NumPy},
	description={\say{El paquete fundamental para la computación científica en Python}, de acuerdo con su \href{https://numpy.org/}{página oficial}. Librería de Python que añade soporte para arreglos y matrices de gran tamaño, junto con una colección de funciones matemáticas para operarlos}
}

\newglossaryentry{Pandas}{
	name={Pandas},
	description={Librería de análisis y manipulación de datos en Python. Más información en su \href{https://pandas.pydata.org/}{página oficial}}
}

\newglossaryentry{Polars}{
	name={Polars},
	description={Librería de manipulación de datos escrita en Rust, con una interfaz de Python. Más información en su \href{https://pola.rs/}{página oficial}}
}

\newglossaryentry{Pytest}{
	name={Pytest},
	description={Framework de pruebas para Python. Más información en su \href{https://docs.pytest.org/en/stable/}{página oficial}}
}

\newglossaryentry{PEP}{
	text={PEP},
	name={PEP (\textit{Python Enhancement Proposal})},
	description={Documento que propone y describe una nueva característica o modificación en Python. En la \href{https://peps.python.org/pep-0001/}{PEP 1} se define qué es un PEP}
}

\newglossaryentry{Flake8}{
	name={Flake8},
	description={Herramienta de Python que combina varios linters para verificar el cumplimiento de las guías de estilo de Python. Más información en su \href{https://flake8.pycqa.org/en/latest/}{documentación}}
}

\newglossaryentry{isort}{
	name={isort},
	description={Librería o utilidad de Python que alfabetiza las importaciones en el código fuente y las separa por tipo, de acuerdo con su \href{https://pycqa.github.io/isort/}{página de web}}
}

\newglossaryentry{black}{
	name={black},
	description={Librería o utilidad de Python que formatea automáticamente el código fuente usando un estilo consistente con cualquier otro código fuente formateado con black, de acuerdo con su \href{https://black.readthedocs.io/en/stable/}{página de web}}
}

\newglossaryentry{doctest}{
	name={doctest},
	description={Módulo en Python que permite probar fragmentos de código incluidos en los \glslink{docstring}{docstrings}, validando que la salida coincida con los ejemplos dados}
}