\chapter{Introducción}

\section{Contexto}

La Universidad de los Andes es una institución educativa de gran renombre, reconocida por estar a la vanguardia de la educación en Colombia, su permanente excelencia académica y su férreo compromiso con la formación integral de sus estudiantes. En aras de propulsar el éxito de sus estudiantes en todo ámbito, la Universidad realiza un esfuerzo constante por ofrecer servicios y recursos que fomenten el desarrollo académico, social, personal y profesional de los alumnos, así como el bienestar de la comunidad universitaria en general.

\subsection{La plataforma \textit{No estás solo}}

En los últimos dos años, la Vicedecanatura de Asuntos Estudiantiles de la Universidad de los Andes identificó dos dolencias significativas concernientes al estudiantado y al profesorado, respectivamente. La primera, consiste en el desconocimiento por parte de los estudiantes de la inmensa cantidad de recursos y servicios de apoyo que la Universidad pone a su disposición en pos de su progreso académico, social, personal y profesional. La segunda, radica en la dificultad y falta de recursos que tenían los profesores a la hora de ejercer su labor como consejeros en apoyo a los estudiantes.

Como herramienta para subsanar estas dolencias, la Vicedecanatura ha impulsado el desarrollo y la extensión de una aplicación web: \textit{No estás solx} (en adelante, \gls{NES}). Esta plataforma fue concebida por la Vicedecanatura de Asuntos Estudiantiles de la Facultad de Ingeniería como una solución integral para garantizar que todos los miembros de la comunidad, especialmente los estudiantes, puedan acceder de manera centralizada y clara a todos los recursos y servicios disponibles.

Desde la perspectiva del estudiantado, \gls{NES} centraliza el acceso a todas las herramientas que fomentan su éxito académico, personal y profesional. Esto incluye información sobre eventos académicos y culturales, procesos y solicitudes administrativas, y servicios de apoyo como consejerías y tutorías. Sumado a eso, facilita el acceso de cada estudiante a su red de apoyo, compuesta por profesores consejeros, coordinadores académicos y otros profesionales de la Universidad. Toda esta información se encontraba dispersa en las distintas plataformas de la Universidad, lo que dificultaba su acceso y su uso conjunto. Esto suponía un reto importante para los estudiantes, quienes debían navegar entre múltiples sistemas y portales para encontrar la información que necesitaban, además de constituir una barrera de entrada significativa para los estudiantes menos experimentados.

Para los profesores y otros usuarios administrativos, \gls{NES} constituye una herramienta que facilita su labor como consejeros y mejora su capacidad de toma de decisiones. La plataforma incluye una base de preguntas frecuentes y guías para poder orientar a los estudiantes en dificultades académicas, personales o financieras. Finalmente, para los coordinadores académicos, \gls{NES} contribuye a descongestionar las solicitudes frecuentes poniendo a disposición herramientas de autogestión que permiten que los estudiantes resuelvan problemas simples por sí mismos, lo cual optimiza el uso del tiempo en las coordinaciones.

\subsection{La necesidad del Perfil del estudiante}

\gls{NES} representa un avance significativo en la centralización y claridad de los recursos y servicios ofrecidos por la Universidad. Sin embargo, previo a la implementación del Perfil del estudiante, los profesores y administrativos enfrentaban desafíos al intentar ofrecer consejerías personalizadas o tomar decisiones informadas. La información sobre los estudiantes estaba dispersa en múltiples sistemas y formatos, lo que dificultaba tener una visión integral de su contexto académico y socioeconómico de forma oportuna.

El Perfil del estudiante surge como una respuesta directa a esta necesidad. Este módulo se integra dentro de \gls{NES} para ofrecer una herramienta que:
\begin{itemize}
	\item Centralice y organice la información relevante sobre los estudiantes.
	\item Presente esta información de manera intuitiva, clara e inmediatamente accesible, sin sacrificar la exhaustividad.
	\item Facilite a profesores y administrativos brindar consejerías personalizadas y tomar decisiones basadas en datos.
\end{itemize}

Con estas características, el Perfil del estudiante no solo mejora la experiencia de los usuarios de \gls{NES}, sino que también fortalece su capacidad de impacto al abordar de manera directa las problemáticas identificadas por la Vicedecanatura de Asuntos Estudiantiles.

\section{Objetivos}

El objetivo general de este proyecto es construir el Perfil del estudiante.

Más específicamente, el objetivo radica en proporcionar a todo el estudiantado y al profesorado de la Universidad de los Andes una herramienta digital que les permita consultar información relacionada con el desempeño académico actual y pasado, así como con el contexto socioeconómico, de los estudiantes de la Universidad.

\subsection{Objetivos específicos}

El objetivo general enunciado recién se desglosa en los siguientes objetivos específicos:
\begin{itemize}
	\item Desarrollar una herramienta de software que permita acceder a la información académica de los estudiantes de la Universidad de los Andes, permitiendo su interacción con una aplicación web, preferiblemente basada en los lineamientos de un \gls{API REST}.
	\item Diseñar e implementar un módulo de la aplicación web \gls{NES} que extraiga y exhiba la información académica de cualquier estudiante de la Universidad de los Andes. Realizar la distinción entre la forma en la que se presenta la información para estudiantes de pregrado y estudiantes de posgrado, en particular, distinguir entre los resultados de un mismo estudiante en su pregrado y en su posgrado, teniendo completa trazabilidad de su trayectoria académica cuando sea el caso.
	\item Conectar el módulo con la navegación y funcionalidades ya existentes en la aplicación web \gls{NES}, de forma que cada estudiante pueda tener acceso a su información académica (mas no a la de otros estudiantes), cada profesor consejero pueda tener acceso a la información académica de sus estudiantes aconsejados y otros usuarios específicos, determinados por la Vicedecanatura, puedan tener acceso a la información académica de los estudiantes que les conciernan.
	\item Garantizar la integridad y seguridad de la información académica de los estudiantes de la Universidad de los Andes, tanto informáticamente como en su presentación, de forma que se cumpla con la normativa de protección de datos personales y de información académica de la Universidad.
\end{itemize}

\subsection{Resultados esperados}

El presente proyecto se puede dar por culminado una vez se satisfaga el siguiente conjunto de resultados:
\begin{itemize}
	\item Existencia de un módulo de la aplicación web \gls{NES} que permita a cada estudiante de la Universidad de los Andes, tanto de pregrado como de posgrado, consultar información relacionada con su desempeño académico.
	\item Existencia de un módulo de la aplicación web \gls{NES} que permita a cada profesor de la Universidad de los Andes consultar información relacionada con el desempeño académico de cada uno de sus estudiantes aconsejados, de forma individual.
	\item Existencia de un módulo de la aplicación web \gls{NES} que permita a usuarios específicos, como decanos, directores de programas u otros directivos de la Universidad de los Andes, consultar información relacionada con el desempeño académico de los estudiantes de la Universidad que les conciernan.
\end{itemize}

\section{Arquitectura general del sistema}

El Perfil del estudiante se construye sobre una arquitectura de software \text{three-tier}, dividida en tres capas: la capa de datos, la capa de lógica y la capa de presentación. Los próximos tres capítulos se centran cada uno en una de estas capas. Esta organización modular facilita el desarrollo, mantenimiento y escalabilidad del sistema, porque permite una separación clara de responsabilidades.

La capa de datos es el cimiento del sistema, encargándose de la extracción, transformación y carga (\gls{ETL}) de los datos. Esta capa garantiza que la información proveniente de diversas fuentes sea estructurada y almacenada de manera consistente, preparándola para ser utilizada por las capas superiores (e, incluso, por otras dependencias de la Universidad para análisis de otros tipos). En esta etapa, los datos fluyen desde sus fuentes originales hasta ser almacenados clasificados semánticamente, como se describe detalladamente en el Capítulo \ref{ch:datos}.

La capa de lógica, descrita en el Capítulo \ref{ch:logica}, actúa como el cerebro del sistema. Aquí, los datos procesados se transforman en recursos, que son expuestos por una \gls{API REST}. Esta capa se ocupa del cumplimiento de las reglas de negocio, es decir, de las restricciones de la vida real y del contexto de la Universidad que deben ser impuestas a los datos. Además, define la interfaz a través de la cual la capa de presentación (y potencialmente otros servicios web similares) puede interactuar con los datos. La capa de lógica es completamente independiente de la capa de datos; es posible cambiar por completo la implementación de cualquiera de las dos capas, sin que sea necesario modificar la otra, mientras se mantenga la interfaz entre ambas.

Por último, la capa de presentación, descrita en el Capítulo \ref{ch:presentacion}, es la cara visible del sistema para los usuarios finales. Su objetivo principal es proporcionar una experiencia de usuario que sea intuitiva, interactiva y accesible, complementada por un diseño visual atractivo. Implementada mediante tecnologías modernas de desarrollo web, esta capa se encarga de traducir los recursos proporcionados por la API en una interfaz atractiva que permite a los usuarios explorar y aprovechar las funcionalidades del Perfil del estudiante. Al igual que antes, esta capa es independiente de las dos capas inferiores, lo que permite que sea modificada o reemplazada sin afectar el funcionamiento del sistema .

En conjunto, estas tres capas trabajan en armonía, la anterior cimentando la siguiente, para constituir el Perfil del estudiante. Son todas independientes entre sí, están implementadas haciendo uso de tecnologías distintas, y eventualmente pueden ser mantenidas y extendidas por equipos de desarrollo diferentes.