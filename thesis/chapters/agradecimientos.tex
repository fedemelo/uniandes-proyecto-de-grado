\chapter*{Agradecimientos}

Este proyecto fue desarrollado a lo largo del año 2024, mi último año como estudiante de pregrado en la Universidad de los Andes. El proyecto recoge mi trabajo como desarrollador en la Vicedecanatura de Asuntos Estudiantiles de la Facultad de Ingeniería. Sin embargo, da la sensación de ser el cierre de mi recorrido académico en la Universidad. Por esta razón, no quiero limitar estos agradecimientos únicamente a quienes contribuyeron directamente al desarrollo del proyecto, sino extenderlos a todas las personas que, de una u otra manera, han sido imprescindibles en mi formación académica y personal.

El año 2024 fue un periodo de muchos cambios y de trabajo diligente. Además de trabajar en este proyecto, desempeñé el rol de monitor de investigación en el proyecto Cupi2, enfocado en mejorar la enseñanza de programación en la Universidad; trabajé como desarrollador de software en el equipo de analítica de datos de la multinacional canadiense Caseware; y concluí las materias necesarias para graduarme como Ingeniero de Sistemas y Computación. Este arduo esfuerzo se vio recompensado con los mejores resultados académicos de mi trayectoria y me permitió culminar mis estudios con el mejor promedio general acumulado de la Facultad de Ingeniería en los últimos quince años.

Eso, por supuesto, no hubiese sido posible solo. Antes que a nadie, quiero agradecer profundamente a Angélica, William y Sebastián: mi familia. Su crianza amorosa forjó mi persona y su apoyo incondicional ha sido el cimiento de mi vida. A ellos les debo este y todos mis proyectos.

En segundo lugar, debo un enorme agradecimiento a mis amigos más cercanos. Ellos son un pilar fundamental en mi vida y en mi felicidad, contribuyendo constantemente a una ya extensa colección de momentos inolvidables. A David, Santiago y Andrés, y a Laura, Laura, Mariana y Sarah: gracias por estar en mi vida. Sin esos dos conjuntos de personas, el camino no hubiese sido tan agradable.

Sumado a lo anterior, por supuesto debo expresar mi gratitud a todos quienes han invertido su tiempo, recursos y esfuerzo en mi educación. Eso incluye a todos quienes fueron partícipes de mi formación tanto en el Colegio San Carlos como en la Universidad de los Andes. No ignoro el inmenso privilegio que constituye haber estudiado en las mejores instituciones educativas del país y haber contado con mentores verdaderamente excepcionales. Mis agradecimientos más profundos van para Jaime Rueda Moreno y Héctor Carranza Granados, quienes me enseñaron a estudiar; y para Nicolás Rincón Sánchez y Alejandro Arturo Pérez Ramírez, quienes me forzaron a hacerlo con rigor.

También quiero agradecer a algunas personas que confiaron en mí y me ofrecieron oportunidades que resultaron ser puntos de inflexión en mi formación. En particular, a Iván David Salazar Cárdenas, quien apostó por mí en el proyecto Cupi2; y a José Joaquín Bocanegra García, quien despertó mi interés por el desarrollo web y auspició mi primera experiencia como desarrollador de software.

Finalmente, quisiera agradecer a quienes estuvieron directamente involucrados en este proyecto. A Mariana y Nicolás, mis compañeros en el desarrollo de No estás solo; a Catalina, Manuel y (nuevamente) Santiago, cuya pericia en ingeniería de datos hizo posible el desarrollo de la capa de datos del proyecto; a Óscar, siempre pendiente de todo; y, por supuesto, a Marcela Hernández, quien confió en mí, me dio la oportunidad de trabajar en este proyecto y me acompañó en todo el proceso.

A todos los acá mencionados y quienes no lo están, pero que han sido parte de mi vida en los últimos años, gracias por acompañarme en esta travesía y por contribuir a que esta etapa de mi vida sea tan significativa. Los resultados son muy buenos y se deben a ustedes. 
