\chapter*{Agradecimientos}

Mi participación en este proyecto fue meramente fortuita. A inicios de 2024, cuando Nicolás Carvajal y Mariana Ruiz me recomendaron como desarrollador a Marcela Hernández, Vicedecana de Asuntos Estudiantiles de la Facultad de Ingeniería, yo no tenía idea de cuál sería su encomienda. Marcela compartió conmigo su sueño: el Perfil del estudiante. A pesar de que la información del estudiantado existía, estaba dispersa y accederla era tortuoso. La centralización y democratización de esa información entre profesores y administrativos, que entonces parecía un objetivo casi utópico, prometía beneficios inconmensurables: decisiones informadas, fundamentadas en datos, y consejerías verdaderamente personalizadas.

No tomó mucho de parte de Marcela para convencerme de involucrarme cuando vislumbré el impacto del proyecto. En su momento, sin embargo, subestimé la magnitud. Aunque en pocos meses desplegamos la primera versión del Perfil del estudiante, mi compromiso inicial de seis meses se convirtió en un año de trabajo arduo, enfrentando retos técnicos que abarcaron todas las capas del desarrollo de software —datos, lógica de negocio, presentación y despliegue—, y viviendo la gratificación de contribuir a un proyecto que impacta directamente la vida de los estudiantes de la Universidad de los Andes.

Yo creo que era eso último lo que nos mantenía despiertos hasta las cuatro de la mañana, a mí y a Santiago Martínez, quien a partir de agosto de 2024 asumió la responsabilidad de la capa de datos.

Este proyecto recoge, entonces, mi trabajo como desarrollador en la Vicedecanatura durante mi último año de carrera, pero también representa el cierre simbólico de mi recorrido académico en la Universidad. Por esta razón, no quiero limitar estos agradecimientos únicamente a quienes contribuyeron directamente al desarrollo del Perfil del estudiante, sino extenderlos a todas las personas que, de una u otra manera, han sido imprescindibles en mi formación académica y personal.

El año 2024 fue un periodo de muchos cambios y de trabajo diligente. Además de trabajar en este proyecto, desempeñé el rol de monitor de investigación en el proyecto Cupi2, enfocado en mejorar la enseñanza de programación en la Universidad; trabajé como desarrollador de software en el equipo de analítica de datos de la multinacional canadiense Caseware; y completé las materias necesarias para graduarme como Ingeniero de Sistemas y Computación. Este esfuerzo se vio recompensado con los mejores resultados académicos de mi trayectoria y me permitió culminar mis estudios con el Promedio General Acumulado más alto de la Facultad de Ingeniería en los últimos quince años.

Eso, por supuesto, no hubiese sido posible solo. Antes que a nadie, quiero agradecer profundamente a Angélica, William y Sebastián: mi familia. Su crianza amorosa moldeó mi persona y su apoyo incondicional ha sido el cimiento de mis logros. A ellos les debo este y todos mis proyectos.

En segundo lugar, debo un enorme agradecimiento a mis amigos más cercanos. Ellos son un pilar fundamental en mi vida y en mi felicidad, y contribuyen constantemente a una ya extensa colección de momentos inolvidables. A David Fuquen, Santiago Martínez y Andrés Guerrero; y a Laura Ramírez, Laura Córdoba, Mariana Rodríguez y Sarah Pérez: gracias por estar en mi vida. Sin esos dos conjuntos de personas, el camino no hubiese sido tan agradable, por decir lo menos.

Extiendo también mi gratitud a todas las personas que, a lo largo de los años, han invertido su tiempo, conocimiento y esfuerzo en mi educación. Eso incluye a todos quienes fueron partícipes de mi formación tanto en el Colegio San Carlos como en la Universidad de los Andes. No ignoro el inmenso privilegio que constituye haber estudiado en las mejores instituciones educativas del país y haber contado con mentores extraordinarios. Mis agradecimientos más profundos van para Jaime Rueda Moreno y Héctor Carranza Granados, quienes me enseñaron a estudiar; y para Nicolás Rincón Sánchez y Alejandro Arturo Pérez Ramírez, quienes me forzaron a hacerlo rigurosamente.

También quiero destacar a algunas personas que confiaron en mi potencial y me brindaron oportunidades que resultaron ser puntos de inflexión en mi formación. En particular, a Iván David Salazar Cárdenas, quien apostó por mí en el proyecto Cupi2; y a José Joaquín Bocanegra García, quien despertó mi interés por el desarrollo web y auspició mi primera experiencia como desarrollador de software.

Finalmente, quisiera agradecer a mis colegas en la Vicedecanatura. A Mariana y Nicolás, quienes mantienen y extienden el resto de No estás solo; a Catalina, Manuel y (nuevamente) Santiago, cuya pericia en ingeniería de datos hizo posible el desarrollo del proyecto; a Oscar, siempre pendiente de todo; y, por supuesto, a Marcela, quien confió en mí, me dio esta oportunidad única de generar impacto mediante mi conocimiento y me acompañó durante todo el proceso.

A todos los acá mencionados y quienes no lo están, pero que han sido parte de mi vida en los últimos años, gracias por acompañarme y contribuir a que esta etapa de mi vida sea tan significativa. Los resultados son muy buenos y les pertenecen a ustedes.
